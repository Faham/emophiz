% 10 to 15 pages
% talk about various aspects of system's implementation and 
% how it is integrated with the interactive technology
% - talk about the sensors and the fuzzy framework for emotion recognition
% and connect that to the game design elements

\section{Recognising Emotion}
Heart rate (HR), blood pressure, respiration, electrodermal
activity (EDA) and galvanic skin response (GSR), as well
as facial EMG (Electromyography) are of physiological variables 
correlated with various emotions most. Interpreting
physiological measures into emotion state can be deficult,
due to noisy and inaccurate signals, however recent on-going
studies in this area by Mandryk and Atkins ~\cite{mandryk2007fuzzy} presented a
method to continuously identifying emotional states of the
user while playing a computer game. Using the dimentional
emotion model and the fuzzy logic, based on a set of physiologcial 
measures, in its first phase, their fuzzy model transforms 
GSR, HR, facial EMG (for fowning and smiling) into
arousal and valence variables. In the second phase another
fuzzy logic model is used to transform arousal and valence
variables into five basic emotion states including: boredom,
challenge, excitement, frustration and fun. Their study successfully 
revealed self-reported emotion states for fun, boredom and excitement 
are following the trends generated by
their fuzzy transformation. The advantage of continiously
and quantitatively assessing user's emotional state during an
entire play by their fuzzy logic model is what makes their
model perfect to be in incorporated with real-time play technologies.
Therefore exposeing user's emotional state as a
new class of uncontious inputs to the play technology.

\section{Affect Engine}
This project has developed a real-time emotion detection
system which can continiously detect and recognise user's
emotional state. The system uses Blood Volume Pulse (BVP),
Galvanic Skin Response (GSR) and Electromyography (EMG;
for frowning and smiling), to classify human affective states
in 2-dimensional valence/arousal space, Figure 2.
The system has three modules, Figure 4:
The Blood Volume Pulse (BVP) signal is a relative measure
of the amount of blood 
owing in a vessel. From BVP we
calculated heart rate and heart rate variability. The heart
rate is known to re
ect emotional activity and has been used
to differentiate between both negative and positive emotions
as well as different arousal levels ~\cite{tt2013procomp}
The Galvanic Skin Response (GSR) sensor to measure the
skin's conductance (between two electrodes and is a function
of sweat gland activity and the skin's pore size). As a person
becomes more or less stressed, the skin's conductance
increases or decreases proportionally ~\cite{picard2003affective}.

The sensor module includes a Thought Technology ProComp Infinity 
encoder ~\cite{tt2013procomp} connected to PC with a USB cable.
SensorLib is the basic module which receives and filters raw
physiological inputs. Then filtered signals are fuzzified by
the use of 22 fuzzy rules in the first phase of transformation.
Then generated arousal and valence values are transformed into
emotion values using another 67 fuzzy rules in
the second pass ~\cite{mandryk2007fuzzy}. Appliactions such as games can easily 
integrate the system where emotion recognition can offer
adaptive control to maintain user interest and engagement.
Once connected via sensors to the emotion recognition system, 
the affective state of the user can be captured continuously and
in real-time and it can be monitored on a displayed
2-dimenstional graph of valance and arousal, Figure 1.

\section{Inegration With Valve Source Engine}
% talk about Modding Half Life EP2
\subsection{Level Design}
\subsection{The Director}
