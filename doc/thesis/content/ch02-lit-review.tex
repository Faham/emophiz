%8 to 20 pages around 50 to 70 ref
%More than a literature review
%Organize related work - impose structure
%Be clear as to how previous work being described relates to your own.
%The reader should not be left wondering why you've described something!!
%Critique the existing work - Where is it strong where is it weak? What are the unreasonable/undesirable assumptions?
%Identify opportunities for more research (i.e., your thesis) Are there unaddressed, or more important related topics?
%After reading this chapter, one should understand the motivation for and importance of your thesis
%You should clearly and precisely define all of the key concepts dealt with in the rest of the thesis, and teach the 
%reader what s/he needs to know to understand the rest of the thesis.

Using emotional responses to increase the level of users interaction with a real-time 
play technology  requires an effective technique to identifying specific emotion 
states within an emotional space. Major existing emotion models in the
psychology literture includes: basic emotion theory ~\cite{ekman1992argument, ekman1992there} ,
dimensional emotion theory ~\cite{lang1995emotion, russell1980circumplex} and models from appraisal 
theory (e.g.,~\cite{roseman2001model}) ~\cite{zhang2010service}

Basic emotion theory identifes anger, disgust, fear, happiness, sadness, and 
surprise ~\cite{peter2006emotion} as the consice set of primary
emotions. These are actually the least six universal categories researchers agreed 
upon ~\cite{zagalo2004story}. It also claims these
primary emotions are distinguishable from each other and
other affective phenomena ~\cite{dalgleish1999handbook}. On the other hand dimensional 
emotion theory argues that all emotional states reside
in a two-dimensional space, defined by arousal and valence.

While there are various opinions on identifying emotional
states, classifcation into discrete emotions ~\cite{dalgleish1999handbook}, or locating
emotions along multiple axes ~\cite{russell1989affect, lang1995emotion}, both 
had limited success in using physiology to identify 
emotional states ~\cite{cacioppo2000psychophysiology}.

Lang used a 2-D space defined by arousal and valence (pleasure) (AV space) 
to classify emotions ~\cite{lang1995emotion}. Valence can be
described as a subjective feeling of pleasantness or unpleasantness while 
arousal is the subjective state feeling activated
or deactivated ~\cite{barrett1998discrete}. Using an arousal-valence space to create
the Affect Grid, Russell believed that arousal and valence
are cognitive dimentions of individual emotion states. Affect is a broad definition 
that includes feelings, moods, sentiments etc. and is commonly used to define the concept of
emotion ~\cite{picard2003affective}. Russell's circumplex model has two "axes"
that might be labeled as displeasure/pleasure (horizontal
axis) and low/high arousal (vertical axis) It is not easy to
map affective states into distinctive emotional states, However these models can 
provide a mapping between predefined
states and the level of arousal and valence ~\cite{zagalo2004story}, Figure 1.

Figure 1: Russell's circumplex model with two axes
of arousal and valence 1

Both mentioned models for identifying emotions convey some
practical issues in emotion measurement. In a HCI context, the 
stimuli for potential emotions may vary less than
in human-human interaction (e.g., participant verbal expressions and body language) 
~\cite{zhang2010service} and also the combination of
evoked emotions ~\cite{peter2006emotion}. However with help of physiological
signals and the fuzzy logic in the model we are going to
use, such issues with our dimentional emotion models would
be minimized. Though it is anticipated to observe different range 
of evoked emotions while interacting with play
technologies compared to interacting with other humans in
daily life. ~\cite{zhang2010service}. However our dimensional emotion models
suffers some other problems. One problems is that arousal
and valence are not independent and one can impact the
other ~\cite{mandryk2007fuzzy}. Continuously capturing emotional experiences
in this applied setting is of its other halmarks. Subjective
measures based on dimensional emotion theory, such as the
Affect Grid ~\cite{russell1989affect} and the Self-Assessment 
Manikin ~\cite{bradley1994measuring}, allow
for quick assessments of user emotional experiences but they
may aggregate responses over the course of many events ~\cite{zhang2010service}.
This work uses Mandryk et al. version of AV space ~\cite{mandryk2007fuzzy}.
