------------------------
People play games to change or structure their internal experiences. Adults in this study, 
enjoy filling their heads with thoughts and emotions unrelated to work or school, 
others enjoy the challenge and chance to test their abilities. Games offer an efficiency 
and order in playing that they want in life. They value the sensations from doing new 
things such as dirt-bike racing or flying, that they otherwise lack the skills, 
resources, or social permission to do. A few like to escape the real world; others 
enjoy escaping its social norms. Nearly all enjoy the feeling of challenge and complete 
absorption. The exciting and relaxing effects of games is very appealing and some apply 
its therapeutic benefits to "get perspective," calm down after a hard day, or build 
self-esteem. Direct observation reveals details about player emotion. We find emotion 
in player's visceral, behavioral, cognitive, and social responses to games. Players play 
to experience these body sensations that result from and drive their actions. Some crave 
the increased heart rate of excitement from a race, the skin prickling sensation from 
Wonder, or the tension of Frustration followed by feelings of Fiero. For others it is 
simply the exchange of worries and thought and feelings for relaxation and contentment 
or a feeling of achievement knowing they did it right.
------------------------
To create more emotion in innovative future games, we at XEODesign want to know more
about the role of emotion in games and identify ways to create emotion other than story 
cutscenes. In improving over 40 million Player Experiences during twelve years of research and
design we have seen people get angry, excited, and on occasion even cry. These reactions
make us wonder how many emotions do games create? What makes failing 80% of the time
fun? Do people play to feel emotions as well as challenge? If emotions are important to play,
where do they come from? Do people modify games to feel differently? Is it possible to build
emotions into games by adding emotion-producing objects or actions to game play rather than
cut scenes? To what extent are game developers already doing this?
------------------------
Pioneers in Player Experience Research and Design methods XEODesign conducted an
independent cross-genre research study on why people play games and identified over thirty
emotions coming from gameplay rather than story. Our results revealed that people play
games not so much for the game itself as for the experience the game creates: an adrenaline
rush, a vicarious adventure, a mental challenge; or the structure games provide, such as a
moment of solitude or the company of friends. People play games to create moment-tomoment 
experiences, whether they are overcoming a difficult game challenge, seeking relief from 
every-day worries, or pursuing what Hal Barwood calls simply "the joy of figuring it out."
------------------------
The Four Keys unlock emotion with:
1. Hard Fun: Players like the opportunities for challenge, strategy, and problem
solving. Their comments focus on the game's challenge and strategic thinking and
problem solving. This "Hard Fun" frequently generates emotions and experiences of
Frustration, and Fiero.
2. Easy Fun: Players enjoy intrigue and curiosity. Players become immersed in games
when it absorbs their complete attention, or when it takes them on an exciting
adventure. These Immersive game aspects are "Easy Fun" and generate emotions
and experiences of Wonder, Awe, and Mystery.
3. Altered States: Players treasure the enjoyment from their internal experiences in
reaction to the visceral, behavior, cognitive, and social properties. These players
play for internal sensations such as Excitement or Relief from their thoughts and
feelings.
4. The People Factor: Players use games as mechanisms for social experiences.
These players enjoy the emotions of Amusement, Schadenfreude, and Naches
coming fromthe social experiences of competition, teamwork, as well as opportunity
for social bonding and personal recognition that comes from playing with others.
------------------------
1. Hard Fun
Emotions from Meaningful Challenges, Strategies, and Puzzles
For many players overcoming obstacles is why they play. Hard Fun creates emotion by
structuring experience towards the pursuit of a goal. The challenge focuses attention and
rewards progress to create emotions such as Frustration and Fiero (an Italian word for
personal triumph). [2] It inspires creativity in the development and application of strategies. It
rewards the player with feedback on progress and success. Players using this Key play to test
their skills, and feel accomplishment. In our study players who enjoy the Hard Fun of
Challenge say they like:
• Playing to see how good I really am
• Playing to beat the game
• Having multiple objectives
• Requiring strategy rather than luck
Games with this Key offer compelling challenges with a choice of strategies. They balance
game difficulty with player skill through levels, player progress, or player controls. In Mario Kart
the difficulty of the challenge matches the skill of novice and advanced players (if you can't
drive, you can at least throw stuff); plus it offers emotion opportunities from cooperative and
competitive gameplay. Games with this Key include Civilization, Halo, Top Spin Tennis,
Crosswords, Hearts, Tetris, and Collapse. Some games offer a choice of winning conditions
such as EverQuest and The Sims.
------------------------
