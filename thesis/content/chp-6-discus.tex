%5 to 10 pages
%talk for the discussion
%State what you've done and what you've found
%Summarize contributions (achievements and impact)
%Outline open issues/directions for future work

The results of our design probe show that adapting the game resulted in higher arousal, but that not all methods were equally effective. In this section, we discuss how game developers and designers can apply our results, consider the limitations of our work, and present the opportunities for future research in this area.

\section{Applying the results}
Our work suggests that adapting games based on a user’s affective state can increase player arousal (excitement) and can potentially automate balancing the difficulty of the game with the affective state of the player. By increasing the challenge of the game when players are not aroused, we can personalize the game experience, drawing the player in.  Conversely, by decreasing the challenge when players feel overwhelmed (too aroused), we can keep the game difficulty manageable and maintain player engagement.

Our work aims to investigate how to adapt games based on a player’s affective state – with the goal of keeping players optimally engaged with the system. Previous work has examined dynamic difficulty adjustment (DDA) for the purposes of balancing multiplayer game play (e.g., [4]). Previous research has shown that when multiplayer games are unbalanced (i.e., one player is much stronger than another), players do not have as much fun [35], and thus there is a need to provide assistance to one player (or hindrance to another) to better balance play. Different approaches have been used to adjust difficulty for player balancing (see [4] and [35]); however, research has not systematically examined whether adjusting the abilities of the player, enemy, or environment affects game enjoyment or player perception. Our work suggests that these different approaches change player experience and thus there is an opportunity to extend our work into the domain of DDA for balancing multiplayer games.

\section{Why adapting the NPC enemy reduced enjoyment}
Our results suggest that helping the player or changing the environment to better support the player are better adaptation approaches than adapting the strength of the NPC enemies. Although a common approach in many games, reducing the difficulty by making the enemies easier to beat resulted in fewer zombie kills (as there were fewer zombies available to kill). This reduction in challenge may have resulted in lower ratings of perceived competence, which in turn reduced players’ enjoyment in the NPC condition.

Self-determination theory [30] suggests that we strive to master challenges, and that this mastery over challenges creates a perception of competence – which is one of our basic needs that must be satisfied for well-being (along with the need for autonomy and need for relatedness).  In the context of games, mastering challenges leads to competence, which ultimately leads to game enjoyment [31]. By adapting the NPC enemies, we give the player less of an opportunity to conquer a challenge, and thus less opportunity to experience competence (and as a result enjoyment). This approach thwarts players from satisfying their needs. Conversely, giving the player enhanced abilities or adapting the environment to support the player in their quest does not seem to negatively affect perceived competence. Adapting the spawn rate or value of helpful items (such as the grenade in our Player condition or the health pack in our Environment condition) does not seem to reduce experienced competence, but allows players to feel like they are achieving in the context of the game.

Recent research in violent imagery in games and the resulting aggression that players experience has suggested that impeding competence in video games fosters aggressive thoughts, regardless of the presence or absence of violent imagery [27]. The authors show how manipulating competence (through manipulating frustrating and complex control schemes, levels of player experience, or game challenges) thwarts need satisfaction amongst players, and increases their access to aggressive thoughts. Although the domain of evaluation (aggressive thoughts) is distinct from our goals, the hypothesis that impeding competence in games thwarts satisfaction of this basic need helps to explain why giving players less challenge to master (as in the NPC condition) does not work as well as giving players the tools and support needed to master greater challenges, as in the Player and Environment conditions.

\section{Limitations and future work}
This design probe represents preliminary work into the domain of affectively-adapting games. There are several limitations in our work that present opportunities for future research. First, the number of participants that we included in our design probe is low (n=16).  Conducting a large-scale experiment would increase the power of our experiment and could reveal differences between the approaches or strengthen existing differences (e.g., the planned contrasts). Second, we investigated the adaptation in a single game genre (FPS game) with specific approaches (e.g., manipulating speed and weapons). Investigating whether our results hold in a different genre or with different adaptation choices would help to generalize our findings. Third, we only adapted based on a player’s galvanic skin response. Using a more sophisticated model that included signals to access player valence (e.g., [21]) would qualify the player’s arousal as either positive or negative in nature. Finally, as noted previously in the discussion, we could consider applying our approach of adaptation based on performance variables, rather than player affect, to examine DDA for the purpose of balancing multiplayer games.
