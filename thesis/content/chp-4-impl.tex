% 10 to 15 pages
% talk about various aspects of system's implementation and
% how it is integrated with the interactive technology
% - talk about the sensors and the fuzzy framework for emotion recognition
% and connect that to the game design elements

In This chapter various aspects of the affect engine developed and used in our study is demonstrated. We would show how the generic design of this system can be incorporated with any game engine and how can it be expanded for any other type of sensor and biofeedback data not necessarily used in this work. The first section talks about the overal design and different modules of the affect engine; Next sections describe different modules in details giving examples of different settings used for our particular study. In final sections we would talk about the game engine we used in this work, and how we incorporated the affect engine in this particular case. We would talk about details of design decisions we made for our study, and game design differences we plan to investigate in our study in Chapter ~\ref{chap:exprm}.

\section{Emotionally Adaptive Game System Design}
%% talk about the fuzzy logic and details of its transformation

We will now present a basic system schematic of an emotionally adapted game in Figure ~\ref{fig:system-design}. A typical game engine depicted on the left-hand side of the diagram, continuously captures user input which is usually collected using gaming controllers such as gamepads or mouse and keyboard.

% Emotionally Adapted Games - An Example of a First Person Shooter.pdf
p We will now present a basic system schematic of an emotionally adapted game in Figure 1. The process of a typical gaming engine is depicted on the left-hand side of the diagram. The engine continuously monitors user input, which is typically collected using a keyboard, a joystick, or other game controllers. This input data is then processed and transferred to the layer that handles the game’s internal logical state, and the user input may influence the game state. After the logical state of the game is defined the system alters the actions of the synthetic agents in the game world. For example, these include the actions of computer-controlled non-player characters. The complexity of this AI layer varies greatly depending on the game. Based on the game state and the determined actions of the synthetic agents, the physics engine determines the kinetic movements of different objects within  the world. Finally, the game world is synthesized for the player by rendering the graphical elements and producing and controlling the audio elements within the game. [see 14] The proposed emotional regulation can be implemented as a middleware system that runs parallel to the actual game engine. The input processing layer of the game engine can receive a data flow of captured and pre-processed sensor data. The real- time signal processing may consist of different forms of amplifying, filtering and feature selection on the psychophysiological signals. This data flow may directly influence the state of the game world, or it can be used by the emotional regulation sub-module of the emotion feedback engine. This module consists of the rules of emotional balancing for different player profile types and gamer-related explicitly set preferences controlled by the “emotion knob”. In addition, it contains a collection of design rules for narrative constructions and game object presentation within the game world. The emotional regulation module also receives input from the game engine’s logical layer to make selections related to desired emotional balance and narrative structures within the game. [14] The outputs of emotional regulation engine may then be applied to various different levels of the actions of the game engine: i) the logical state of the world may be re-directed, ii) the actions of the synthetic agents may be controlled, iii) the 

A basic system schematic of an emotionally adapted game is presented in Figure ~\ref{fig:}

\img
{Emotion adaptive game system design}
{Emotion adaptive game system design}
{system-design.pdf}
{system-design}


% Emotionally Adapted Games - An Example of a First Person Shooter.pdf
p kinetics of the game may be altered and iv) the rendering of the game world may be changed. First two options are more relevant to high-level and story-related structures of the game, whereas the last two are more directly related to the selection of presentation of objects within the virtual environment. [e.g. 14]

% Emotionally Adapted Games - An Example of a First Person Shooter.pdf
p With our system design for games it is possible for the game designer as well for the user to set desired emotional targets to be approached or avoided. The system uses both positive and negative feedback loops to determine the ideal adaptations case-by- case for game play for various emotional effects to be realized and managed. Indeed, to implement and evaluate some of the ideas presented, we have explored novel technical solutions and tested different kinds of psychophysiological adaptations that can be implemented. EMOShooter is a prototype platform for psychophysiologically adaptive 3D first-person shooter (FPS) gaming. It is built on top of open-source graphics engine (OGRE 3D) and physics engine (ODE). In this experimental platform we have the possibility to modify practically any game world element, player avatar, avatar outlook, or control parameter.  EMOShooter is a simple psychophysiologically adaptive game and hence a part of our emotionally adapted games definition. The system uses psychophysiological signals to influence the ease of use of the controls of the game hence affecting game play difficulty and game play experience. The system does not have target experiences systematically implemented at this moment nor does it have an emotion knob to tune the system. However, the EMOShooter game is a valuable example of one type of emotionally adapted games in demonstrating one feasible link between real-time emotional state measurement with psychophysiology and the game play. The goal of the EMOShooter game is to kill cube-like enemies either with sniper or machine gun. We have been testing various adaptation patterns with EMOShooter by primarily EDA and respiration as psychophysiological signals in our adaptive 

% Emotionally Adapted Games - An Example of a First Person Shooter.pdf
p feedback system regards how these signals can be meaningfully connected to the actual game play via adapting game controls.  Adaptation of game controls includes changes in rate of fire, recoil, movement speed and shaking. If a player is aroused this will be reflected in EDA and respiration signals which in turn will make rate of fire and movement slower and will make the aim shaky. Hence, for a highly aroused player the game becomes more difficult. For a mildly aroused or calm player the controls become more efficient and easy to use hence facilitating performance at game play. Game events are mostly arousing. The amount of cubes to shoot, their approach and firing on the user, the amount of health left after being hit and the sound effects all are geared to drive up arousal in the game. The player’s task is to be calm as indexed by psychophysiological signals to be able to operate the controls more efficiently. In our tests of the game we have collected also EMG data to infer the valence dimension of emotion during game play. In addition to the psychophysiological signals we have collected data from the players using behavioral game logging, video capture, interviews and questionnaires. During our tests we noticed that proper calibration and base lining of the psychophysiological signals is very important for the adaptations to work. We also noticed that having robust stimuli in the game is crucial for the adaptations to work because in many cases the stimulus functioned as a trigger in adaptation. The psychophysiological signals used are calibrated by using dynamic range (basically a variation of dynamic signal normalization algorithm), which has a memory buffer of a few seconds (depending on signal). Dynamic range is easy to use and effective calibration mechanism, and relative change seems to be more practical than absolute values in this kind of gaming.  According to our early analysis, there are three key issues in designing psychophysiologically adaptive games i) understanding the meaningful emotionally adaptive gaming patterns, ii) implementation of adaptation algorithms and signal processing, and iii) purposeful use of sensors in the game context [15]. The design patterns used in emotionally adaptive gaming must be meaningful and enjoyable for the player, and the utilization of signals must also obey the overall goal of the game. In order to achieve the goal player should find the right rhythm or balance of playing the game and control of psychophysiological responses and signals.   Signals should be analyzed as close to real-time as possible in psycho- physiologically adaptive gaming in order to keep the feedback loop in pace with the game adaptations and game events. We have used time-series analysis with short sample windows. In practice, ECG, EEG and EMG always require extensive data processing, but EDA and respiration can be almost used as such to create the adaptation signal. This implies that not all psychophysiological signals are equally open to be used as real-time inputs into an adaptive game at least in this stage of signal processing hardware and software development. Usability of psychophysiological recording devices remains quite poor. Respiration, HR [heart rate] and EDA are probably the easiest to implement. Also in case of emotional adaptation the design of the game may include the physical design of the sensors, e.g. “Detective hat” for EEG sensors or “Sniper-gloves” for EDA sensors. Hence, the sensors could be designed as part of the game story rather than presented as cumbersome and invasive laboratory-originated equipment. 

% Emotionally Adapted Games - An Example of a First Person Shooter.pdf
p In future versions of EMOShooter we may also employ the system design of emotionally adapted games including setting of explicit experiential targets and their parameters for gaming sessions and the emotion control knob.


\subsection{Recognizing Emotion} %revision: 0
Heart rate (HR), blood pressure, respiration, electrodermal activity (EDA) and galvanic skin response (GSR), as well as facial EMG (Electromyography) are of physiological variables correlated with various emotions most. Interpreting physiological measures into emotion state can be difficult, due to noisy and inaccurate signals, however recent on-going studies in this area by Mandryk and Atkins ~\cite{mandryk2007fuzzy} presented a method to continuously identifying emotional states of the user while playing a computer game. Using the dimensional emotion model and the fuzzy logic, based on a set of physiological measures, in its first phase, their fuzzy model transforms GSR, HR, facial EMG (for frowning and smiling) into arousal and valence variables. In the second phase another fuzzy logic model is used to transform arousal and valence variables into five basic emotion states including: boredom, challenge, excitement, frustration and fun. Their study successfully revealed self-reported emotion states for fun, boredom and excitement are following the trends generated by their fuzzy transformation. The advantage of continuously and quantitatively assessing user's emotional state during an entire play by their fuzzy logic model is what makes their model perfect to be in incorporated with real-time play technologies. Therefore extracting user's emotional state as a new class of unconscious inputs to the play technology.

Using emotional responses to increase the level of users interaction with a real-time play technology requires an effective technique to identifying specific emotion states within an emotional space. Major existing emotion models in the psychology literature includes: basic emotion theory ~\cite{ekman1992argument, ekman1992there} , dimensional emotion theory ~\cite{lang1995emotion, russell1980circumplex} and models from appraisal theory (e.g.,~\cite{roseman2001model}) ~\cite{zhang2010service}


\subsection{Fuzzy Logic for Space Transformation}
\subsection{Physiological Signals to Arousal and Valence}

\img
{DotFuzzy Application used to define set of fuzzy rules and store it in xml format}
{DotFuzzy Application}
{dotfuzzy.png}
{dotfuzzy}

\section{Affect Engine} %revision: 2
Affect Engine is the software unit developed to transform collected physiological data to their equivalent emotional state in real-time. While it is generally agreed that emotions comprise three components: subjective experience (e.g. feeling joyous), expressive behavior (e.g. smiling), and physiological activation (e.g. arousal) ~\cite{scherer1993neuroscience}, Affect Engine provides a framework for transformation of physiological activations and some expressive behaviors. Affect Engine consists of four major components: Sensor Module, Fuzzification Module, Administration Panel and Engine Proxies, Figure \ref{fig:affect-engine} is a schematic view of these components working together. Applications such as games can easily integrate the Affect Engine where emotion recognition can offer adaptive control to maintain user interest and engagement. Once connected via sensors to the emotion recognition system, the affective state of the user can be captured continuously and in real-time, and used as a secondary input in the game logic for an enhanced play experienced.

\img
{Affect engine modules}
{Affect engine modules}
{affect-engine.pdf}
{affect-engine}

\img
{Sample connected system}
{Sample connected system with GSR and EMG sensors attached}
{sample-connected-system.png}
{sample-connected-system}

At the following a brief description on these components is provided.

\subsection{Sensor Module} %revision: 3
The sensor module consists of a Thought Technology ProComp Infinity encoder ~\cite{tt2013procomp} Figure ~\ref{fig:encoder}, connected to PC with a USB cable, SensorLib as the basic application programming interface (API) receives raw physiological inputs from the encoder driver and provides functionalities to apply different filters such as low-pass, high-pass, smoothing and shifting to the signal.

\img
{Thought Technology ProComp Infinity Encoder}
{Thought Technology ProComp Infinity Encoder}
{encoder.png}
{encoder}

\subsection{Fuzzification Module} \label{subsec:fuzzi} %revision: 3
This module functions through two separate phases; Then filtered signals are fuzzified using a set of fuzzy rules in the first phase of transformation. Then generated arousal and valence values are transformed into emotion values using another set of fuzzy rules in the second pass ~\cite{mandryk2007fuzzy}. A sample set for fuzzy rules used in the first and the second phase can be found in Appendix ~\ref{app:phys-to-av} and ~\ref{app:av-to-emotion}.



\subsection{Administration Panel}

\img
{Administration Panel}
{Administration Panel}
{placeholder.png}
{administration-panel}

\subsection{Engine Proxies}


\section{Sensors}
While the modular design of the Affect Engine allows its expansion for support of any physiological measures, currently the system uses Blood Volume Pulse (BVP), Galvanic Skin Response (GSR) and Electromyography (EMG; for frowning and smiling), to classify human affective states in 2-dimensional valence/arousal space (Figure \ref{fig:russel-av-space}).

\subsection{Blood Volume Pulse and Heart Rate}
The Blood Volume Pulse (BVP, Figure \ref{fig:bvp-sensor}) is a relative measure of the amount of blood owing in a vessel. Using BVP we calculated heart rate and heart rate variability. The heart rate is known to reflect emotional activity and has been used to differentiate between both negative and positive emotions as well as different arousal levels ~\cite{tt2013procomp}.

\subsection{Galvanic Skin Response}
The Galvanic Skin Response (GSR, Figure \ref{fig:gsr-sensor}) is useful to measure the skin's conductance between two electrodes and is a function of sweat gland activity and the skin's pore size. As a person becomes more or less stressed, the skin's conductance increases or decreases proportionally ~\cite{picard2003affective}.

\subsection{Facial Electromyography}


\section{Integration with Valve Source Engine}

\img
{Zombie model that attach player as enemy}
{Zombie model}
{zombie.png}
{zombie}

% talk about Modding Half Life EP2
\subsection{Level Design}

\largeimg
{Hammer level editor from the Source}
{Hammer level editor}
{hammer-editor.png}
{hammer-editor}

\largeimg
{Map of the level designed for study in next chapter}
{Map of the level designed for the study}
{map-of-level.png}
{map-of-level}


\subsection{The Director}

%-----------------------------------------------------------
\begin{center}
\label{tbl:adjustment-strategy}
\captionof{table}{Adjustment Strategy}
\begin{tabular}{+l^p{4cm}^p{4cm}^p{4cm}}
\specialrule{.1em}{.05em}{.05em}
\rowstyle{\bfseries}
Excitement/Strategy               & Player                         & NPC                            & Environment \\
\hline
Excited                           & Increase player speed \newline
                                     Increase grenade rate \newline
                                     Increase med-pack rate         & Decrease zombie speed \newline
                                                                      Decrease zombie crowd          & Decrease fog density \\
\hline
Not excited                       & Decrease player speed \newline
                                     Decrease grenade rate \newline
                                     Decrease med-pack rate         & Increase zombie speed \newline
                                                                      Increase zombie crowd          & Increase fog density \\
\specialrule{.1em}{.05em}{.05em}
\end{tabular}
\end{center}
%-----------------------------------------------------------