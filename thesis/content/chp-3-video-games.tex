% 10 to 15 pages

Playing video games is a kind of experiential entertainment would help people to have new internal experiences. The virtual world of video games let adults to play new roles and experience emotions. Games provide opportunities for the development and design of environments where the player can interactively experience various emotions and mental conditions. This interactive experience is in contrast to cinema and other more passive forms of entertainment.

In computer games, gameplay is usually considered of key importance ~\cite{rollings2006fundamentals, malone1982makes}. One can define gameplay as the pattern defined through the game rules ~\cite{salen2004rules, pajares2008understanding} the connection between the player and the game ~\cite{laramee2002game} or challenges ~\cite{rollings2003andrew} presented by the game. Gameplay is not a singular entity, and can consist of many different elements. Gameplay is essentially a synergy that emerges from the inclusion of certain factors ~\cite{rollings2003andrew}. In absence of a broadly accepted definition for gameplay, my focus here is targeted on one core element \textit{challenge}. The sense of challenge in video games is a significant contribution to continued play. However the challenge gameplay element should be carefully adjusted for the targeted audience. The process of adjusting the challenge level of the game is usually referred to as game balancing. To balance the challenge level or difficulty scale of the game, designers must change many interrelated parameters to create a experience somewhere between too easy and boring and too hard and frustrating ~\cite{koster2013theory}. In this chapter, a history of related works investigating the relation between a game's difficulty level and various emotional states is provided.

\section{Gameplay and The Concept of Flow}
Mihaly Csikszentmihalyi, in the mid 70s, in an attempt to explain happiness, introduced the concept of \textit{flow}. His work as a professor of psychology has become fundamental to the field of positive psychology that includes happiness, creativity, subjective well-being and fun ~\cite{csikszentmihalyi1990flow}. The feeling of complete and energized focus while engaged in an activity is usually referred to as flow, this feeling also has an associated sense of enjoyment and fulfillment ~\cite{csikszentmihalyi1990flow}. During the flow experience my focus maximizes performance and I lose track of time and worries. Flow is also referred to as the optimal experience or being in \textit{the zone}.

Csikszentmihalyi in his work, identified eight major components of flow ~\cite{csikszentmihalyi1990flow}:

\begin{itemize}
\item A challenging activity requiring skill;
\item A merging of action and awareness;
\item Clear goals;
\item Direct, immediate feedback;
\item Concentration on the task at hand;
\item A sense of control;
\item A loss of self-consciousness; and
\item An altered sense of time.
\end{itemize}

An activity doesn't necessarily require all the eight components to inspire the flow experience. I will constrain my analysis to the first item which relates to the challenge and the skill level. Figure ~\ref{fig:challenge-vs-skill} shows Csikszentmihalyi's flow model in terms of challenge and skill.

\img
{Challenge and skill level}
{Mental state in terms of challenge level and skill level, according to Csikszentmihalyi's flow model ~\cite{csikszentmihalyi1997finding}}
{challenge-vs-skill.pdf}
{challenge-vs-skill}

Although there are many components that go into a great player experience, games at their core motivate players by giving them the opportunity to demonstrate mastery over game challenges ~\cite{ryan2006motivational}. To feel accomplishment over mastering game challenges, designers adapt parameters to create gameplay that resides somewhere between too easy and boring and too hard and frustrating ~\cite{koster2013theory}. The flow zone is a concept in flow theory and is illustrated in Figure ~\ref{fig:flow-zone}. The flow zone suggests that, in order to sustain players' flow experience, designers must balance the inherent challenge of the activity and the required player's ability (skills) to address and overcome it ~\cite{chen2007flow}. Good design avoids the activity becoming so overwhelming a challenge that it generates anxiety, and avoids failing to engage the player, becoming so boring due to a lack of challenge. One can consider the flow zone as a fuzzy area where the activity is not too challenging or boring ~\cite{csikszentmihalyi1990flow}.

\img
{Flow zone}
{Flow zone, the area where challenge and skill level match.}
{flow-zone.pdf}
{flow-zone}

Because playing a video game should gradually increase a player's skill level, the designer should increase the required skill level by changing the challenge level of the game at the same pace to keep the player in flow. However, the rate of skill acquisition varies individually. Designing such a balance between the challenge and skill level becomes a greater and greater challenge for the designer as the size of the targeted audience grows. For example, when designing a game for kids, this balance would have a wholly different rate of change than when designing it for adults.

\section{Dynamic Game Balancing vs. Static Game Balancing}
Many video games offer only a simple, narrow and static experience, which is denoted by a red line in Figure ~\ref{fig:flow-zone}. This statically preset path might keep the typical player in the flow zone but will not be fun for the hardcore or novice player ~\cite{chen2007flow}. For example simple skills for typical players such as walking in a 3D space and looking around by controlling the camera can easily be found new and cumbersome to casual players who are used to 2D games. This potentially frustrating introductory challenge combined with the intended game challenges can make casual gamers turn away. One should note that frustration due to lack of skill during game play is not necessarily same as frustration caused by difficult game levels. In fact, Kiel identified two types of frustration during games, the at-game-frustration and in-game-frustration. The first is due to lack of skill during game playing and the second in caused by difficult game levels ~\cite{gilleade2004using}.

\img
{Adapted flow zone}
{Adapted flow zone}
{flow-zone-adapted.pdf}
{adapted-flow-zone}

For many years, game designers aimed to provide some customizations, for example by letting players choose a difficulty level upfront or including progressive difficulty levels during gameplay, based on a player's performance. More advanced methods that work in real-time are less common, as most designers predefine levels of game challenge for players with different skill levels. Another approach to address this balance issue is by techniques known as rubber-band artificial intelligence (AI) ~\cite{champandard2003ai}: when falling behind, the player suddenly gets extra help, which allows for catching up again (and vice versa for the opponents).

Designers work on many different aspects of the game to make it balanced. Game balancing in terms of difficulty level and player experience is only one aspect of balancing a game. Another important balancing issue is the concept of fairness in the game. A primary issue in competitive games is that various settings of properties for different characters should have equal chances to win the game based on rules and starting positions ~\cite{rollings2003andrew}. Balancing fairness may involve manipulations of different game elements - for example initial resources and abilities allocated to different player  types like Orcs or Humans in WarCraft. This type of static balancing is often carried out through repeated playtesting of the game mechanics and parameters, such as tuning the capabilities of individual weapons or units ~\cite{boll2003paper, rollings2003andrew}.

\largeimg
{Call of Duty, static difficulty modes}
{Menu content for difficulty selection, Call of Duty: Modern Warfare (Wii)}
{cod-4-mw-wii-select-difficulty.png}
{cod-mw-difficulty}

In computer game development, designing agents whose behavior challenges human players adequately is a key issue. The idea of a dynamically adapting agent behavior or dynamically balancing a game during game play through AI difficulty is not new ~\cite{andrade2005automatic}. Dynamic game balancing (DGB) also known as dynamic difficulty adjustment (DDA) is automatically changing various aspects of a video game in real-time in order to better correlate players abilities to game challenges. These adjustments can happen in different places such as game mechanics, game scenarios or agent behaviors. In DGB games are changed to avoid players become bored (if the game is too easy) or frustrated (if it is too hard) from start to the end. It aims to detect players skill level dynamically and adjust game challenges in accordance to them, while the player is progressing and acquiring new skill in the game. Dynamic balancing differs from static balancing because the interaction of the player or players with the game should be considered, and different units and parameters in the game configuration should be adapted based on the current state of the game ~\cite{tan2011dynamic} rather than at the start of play based on player models. Variable frequency of enemies in Diablo 3 and variable power of enemies in Assassin's Creed 4: Black Flag are examples of dynamic balancing during game play.

Many different approaches are found to address dynamic game balancing. In all cases, it is necessary to measure the difficulty the user is facing during the game, which can happen either implicitly or explicitly. This measure tries to identify the difficulty the user is facing at a given moment. This measure is usually performed by a heuristic function, usually called the challenge function. Given a specific game state this function can specify how easy or difficult the game feels to the user. Many different in-game properties such as the rate of successful shots or hits, the numbers of won and lost pieces, life points or time to complete some task, can be used for this measure.

Huniche et al. ~\cite{hunicke2004ai} controlled the game environment settings in order to increase or decrease the level of challenges. The player is more likely to get more ammunition and life points if the game is too hard. Another straightforward approach is to combine such environmental manipulations with some mechanisms to adapt the behavior of the NPCs or intelligent agents controlled by the computer. This adjustment, however, should be made with moderation.

Using behavior rules is one of most popular traditional implementations of such intelligent agents. For example, in a typical fighting game, a behavior rule would state ``kick the opponent if he is reachable, chase him otherwise". Extending such an approach to include opponent modeling can be made through Spronck et al.'s dynamic scripting ~\cite{spronck2004difficulty} which assigns a probability to each rule. Rule probability weight can be dynamically changed and adjusted through the game according to the opponent skills, leading to adaptation to the specific user. Rules that are neither tool strong nor too weak for the current player can have higher probability to be picked.

\subsection{AI in Dynamic Game Balancing}
Works in the field of DGB is usually based on the hypothesis that interactions between player and opponents, is the component that contributes the majority of the entertainment in a computer game ~\cite{schaalevolving}. In recent years, many high quality games to a rely on high quality AI as an important selling point ~\cite{forbus2002ai}. Xiang et al. in their work on dynamic difficulty adjustment by facial expression ~\cite{xiang2013dynamic} have employed Gaussian Mixture Module and multi variate pattern mining to model the player's reaction pattern ~\cite{lee2006dynamic, chiu2008using}. They have also controlled NPCs behaviors using reinforce learning algorithm ~\cite{spronck2004difficulty, andrade2005challenge}. Hunicke ~\cite{hunicke2004ai} used the Hamlet system to predict when the player is repeatedly entering an undesirable loop, and help them get out of it. They have explored computational and design requirements for a dynamic difficulty adjustment system using probabilistic methods based on the Half Life game engine. Joost ~\cite{westra2009adaptive} proposed an adaptation approach that uses expert knowledge for the adaptation. They used a game adaption model and organized agents to choose the most optimal task for the trainee, given the user model, the game flow, and the capabilities of the agents. Hom ~\cite{hom2007automatic} used AI techniques to design balanced board games like checkers and Go by modifying the rules of the game, not just the rule parameters. Olesen explored neuro-evolution methodologies to generate intelligent opponents in Real-Time Strategy (RTS) games and tried to adapt the challenge generated by the game opponents to match the skill of a player in real-time ~\cite{olesen2008real}.

Demasi and Cruz ~\cite{demasi2003line} developed NPCs employing genetic algorithm techniques to keep alive those agents that best fit the user skill level. Further studies by Yannakakis and Hallam ~\cite{yannakakis2006towards} have shown that artificial neural networks (ANN) and fuzzy neural networks can help to better recognize player satisfaction level, given appropriate estimators of the challenge and curiosity (intrinsic qualitative factors for engaging gameplay according to Malone) ~\cite{malone1982makes} of the game and data on human players' preferences.

\subsection{Dynamic Game Balancing in Recent Games}
In recent years many well known game titles have integrated more complex dynamic game balancing mechanisms. The 2008 video game Left 4 Dead integrated a new AI technology called \textit{The AI Director} ~\cite{left2008dead}. The AI Director monitors individual player's and groups of players' performance and their progress in the game, and how well they work together, and dynamically determines the number of zombies that attack the player, and when boss fights should happen. The director also makes decisions about audiovisual elements of the game to attract players' attention to certain areas or set the mood ~\cite{left4dead2009handson}. This technique, also called \textit{Procedural narrative}, tries to analyze players' experience in the game and control up-coming events to give the player a sense of narrative. In 2009, Resident Evil 5 employed the \textit{Difficulty Scale}. This mechanism, mentioned in the official strategy guide, grades the player's performance on a scale from 1 to 10, and dynamically adjusts NPC behaviors like attacking and enemy strength, damage, and resistance based on the player's performance. Player performance is estimated based on different in-game variables such as, deaths, damage dealt and critical attacks. The statically selected difficulty levels of the game locks players at a certain number; for example, the Normal difficulty, locks player performance at grade 4, but will dynamically change based on a player's performance between 2 (if player is doing poorly) and 7 if doing well ~\cite{resident2009evil}. Fallout: New Vegas and Fallout 3 are of other well known game titles utilizing dynamic difficulty adjustment techniques. In these titles, players would encounter more challenging combatants while progressing in the game. The system is designed to retain a constant difficulty level while the player's skill increases.

Addressing the game balance problem using predefined difficulty levels cannot incorporate the behaviors of all potential players using a player's in-game data and employing artificial intelligence can generates predictable behaviors which reduce the believability of the non-player characters (NPCs). Furthermore, human players enhance their skills while playing a game which necessitates an adaptive mechanism for more challenge during play ~\cite{olesen2008real}. I should also mention, even with all the development of AI in computer games, players often still find playing against human controlled opponents more interesting than computer controlled ones ~\cite{weibel2008playing}.

\section{Emotionally Adaptive Games}
While adjusting the challenge level is crucial to video game design, what styles of game play is appealing differs from person to person. For example skill level differences between different players might make a difficulty level which is enjoyable by a novice, but boring for an expert player; Games therefore need psychological customization techniques ~\cite{saari2005towards}. Game adaptation that is solely based on in-game performance can only have limited success, because it adapts to performance not experience ~\cite{bartle1996hearts}. Each type of player has his/her own goals, preferences and emotional responses when playing a game. To optimize a player's experience, psychological customization requires a game to take the emotional state of the player into account. Games should become emotionally adaptive (Figure ~\ref{fig:affective-game-system}) ~\cite{tijs2009creating}.

Affective computing can have a major impact on not only video games but any form of computing reliant on human interaction. The concept of affective gaming was first introduced by Wehrenberg, through using Biofeedback to control a game based on relaxation level. It was one of the earliest studies on correlating a game with player's biofeedback. After years of research the project was first implemented in 1984 for Apple II computers. The results of that study proved that human arousal level can actually be measured through GSR and employed to control a game ~\cite{wehrenberg1995willball}. Different emotion theories as described in chapter ~\ref{chap:emotion} can be utilized for analysis and estimation of human affect state while interacting with computers. Because a user's affective state can dynamically change from an emotional perspective during an interactive experience, emotional human-computer interaction works in an \textit{affective loop} ~\cite{sundstrom2005user}. Polaine in his work on the flow principle in interactivity ~\cite{polaine2005flow} argues that flow is a feedback loop of action-reaction-interaction and involves collaboration or exchange (with real or computer agents). My work is based on a similar feedback loop in a game context which dynamically adjusts a game's difficulty level by measuring a user's affect state. Figure ~\ref{fig:affective-game-system} ~\cite{tijs2009creating} shows a schematic view of this closed affective loop for an emotionally adaptive game. In this closed loop, by continuously looking at the gamers emotional state the game influences the player's experience and emotional state by providing the right game mechanics ~\cite{hunicke2004mda}. Ideally, during play, the emotional state of the player (measured in terms of emotion-data), is continuously being fed back to the game so that the game can adapt its mechanics (e.g. difficulty level) in real-time, with an eye towards enhancing flow ~\cite{chen2007flow} or immersion ~\cite{nacke2008flow}.

\img
{Emotionally adaptive games}
{The emotionally adaptive game loop, inspired by the affective loop ~\cite{sundstrom2005user}.}
{affective-game-system.pdf}
{affective-game-system}

\subsection{Why and How to Emotionally Adapt Games}
Emotionally adapted gaming can be seen as collection of affective game adaptation decisions which are parts of the meta-narrative of the game ~\cite{saari2009emotionally}. Therefore, an approach to systematically identify and design these adaptations decisions is to base them on psychologically validated templates. Each one of these adaptation elements' influence (such as emotional response) on a particular type of user is sufficiently predictable ~\cite{saari2009emotionally}. These adaptation templates may consist of different game manipulation approaches:

\begin{itemize}
\item Manipulating the substance of a game at its basic level, such as changes in story line and putting the player in different situations.
\item Manipulating the game in presentation level, such as visual elements, shapes, colors, sound effects and background music.
\item Manipulating the game at the interaction level. The difficulty level or challenge level of the game may also be continuously adjusted, keeping the skills and challenges in balance which results in a maintenance of an optimal emotional experience and possibly also a flow state ~\cite{saari2005emotional}.
\end{itemize}

% Emotionally Adapted Games - An Example of a First Person Shooter.pdf(paraphrase)
To manipulate emotions in gaming on the basis of avoiding or approaching a specific emotional state, Saari et al. categorize manipulation goals and strategies to the followings:

\subsubsection{Manipulating Emotions Through Narrative Features} There are the transient basic emotional effects of games that are dependent of the phase of the game or specific events. These are emotions such as happiness, satisfaction, sadness, dissatisfaction, anger, aggression, fear and anxiousness. These emotions are the basis of narrative experiences, i.e. being afraid of the enemy in a shooting game, feeling aggression and wishing to destroy the enemy and feeling satisfaction, even happiness, when the enemy has been destroyed. Emotional regulation systems in these instances most focus on manipulating the event structures, such as characters, their roles, events that take place and other features of the narrative gaming experience ~\cite{saari2005emotional}.

\subsubsection{Eliminating Unwanted Emotion Experiences Through Basic Game Structure} There are possibilities for emotional management, especially in the case of managing arousal, alertness and excitation. One may also wish to manage negative emotions, such as sadness, dissatisfaction, disappointment, anger, aggression, fear and anxiousness. The case for managing these emotions is twofold. On the one hand, one may see that these negative emotions could be eliminated in the gaming experience, by damping the emergence of such emotion in the game. For example, one could make a deliberately happy game with monkeys on a far away island throwing barrels at obstacles to gather points. This would include minimum negative emotions. Or, in a game where negative emotion is a part of the game, one may wish to limit the intensity, duration or frequency of the emotions via manipulating gaming events and gaming elements so that sadness or fear are at their minimum levels, or that gaming events do not lead to sadness at all ~\cite{saari2005emotional}.

Similarly, managing arousal or the intensity, duration and frequency of select negative emotions may be feasible as a form of parental control. On the other hand, one may wish to maximize arousal, alertness and excitation, perhaps even anger, fear and aggression for hardcore gamers.

\subsubsection{Avoiding Unwanted Emotions Emerged From Improper Game Balance By Dynamic Adaptation} There are possibilities related to the avoidance of certain types of emotions that are typically indicative of a poor gaming experience. Inactivity, idleness, passivity, tiredness, boredom, dullness, helplessness as well as a totally neutral experience indicate that there is a fundamental problem in the user-game interaction. This could be due to a poor match between the gaming skills of the user and the challenges of the game or some other factors, such as the user is stuck on a level because it is unclear how to progress. When a gaming engine detects these emotions, it may adapt its behavior to offer the user a different difficulty level or offer the user clues as to how to progress ~\cite{saari2005emotional}.
% Emotionally Adapted Games - An Example of a First Person Shooter.pdf

\section{Related Work}

Previous research attempts to create emotionally adaptive software have mainly focused on tutoring systems and workload / performance optimization (see e.g. ~\cite{schaefer2008usability}). Fewer attempts have been made to incorporate a closed-loop mechanism in a games context. Takahashi et al. ~\cite{takahashi1994experimental} and Rani et al. ~\cite{rani2005maintaining} created a game that was found to improve player performance by adapting difficulty level to player's physiological state. Claims from these both studies were, however, based on a limited number of participants. A number of biofeedback games have recently been developed, which integrate some aspects of a player's physiological data into the game (e.g. ~\cite{bell2003journey}, ~\cite{bersak2001intelligent} and ~\cite{xiang2013dynamic}). These games focus on stress manipulation rather than optimization of gameplay experience. In this section a number of noticeable works related to emotionally adaptive games are introduced and some of their properties, achievements and limitations are investigated.

%Creating an Emotionally Adaptive Game.pdf
\subsection{Emotional State and Unguided Player Speed Variation}
Tijs et al. in their work on emotionally adaptive games have developed a version of the Pacman PC-game (Figure ~\ref{fig:pacman}) called Stimulus ~\cite{tijs2009creating}. They chose Pacman for a number of reasons to conduct their study, (1) relatively uncomplicated nature of the game, which could lead to emotional bias, (2) it is a well-known game and is easy to pick up and requiring relatively short practice to minimize learning effects, and game play (3) because Pacman has a continuous action flow which is beneficial when comparing blocks of game play time. The game has been used in other affective computing studies (e.g. ~\cite{yannakakis2007towards}). In ~\cite{tijs2009creating} the following changes to game play were made: (1) The players played the same level of difficulty during the experiment, (2) Entities that were eaten, such as points and pills, respawned, (3) The speed of the player changed at preset times (unknown to the player), (4) Eating objects increased the player's score but being eaten by the enemies meant a strong decrease in score, and (5) The overall objective of the game was to score as many points as possible. Their choice for manipulating speed as the difficulty parameter, instead of the number of enemies has been due to the fact that the number of normal ghosts changed during the gameplay as a result to Pacman eating star-shaped pills. This game was played using arrow keys on the keyboard ~\cite{tijs2009creating}.

\img
{Pacman}
{Pacman - The original game used by Tijs et al.}
{pacman.png}
{pacman}

Tijs et al. study on Stimulus has shown the unguided adaption of players speed has resulted in the slow-mode being too slow and the fast-mode being too fast for some players. They suggested that the speed level in the normal-mode might not be optimal either, but the players' experiences are better in that mode than in the other two.

They have described their work on induction of boredom, frustration and enjoyment through manipulation of the game mechanic ``speed'' partly successful. Nearly all players indicated boredom during the slow-mode, however, the fast-mode was found more enjoyable than frustrating. As they demonstrated in their work, players knew the game speed was going to change, and also they knew it only lasted for a limited amount of time. Finally, they concluded nearly all participants described the normal-mode as the most enjoyable of the three.

%Affective Gaming - a GSR Based Approach .pdf
\subsection{Emotion and Different Difficulty Levels}
Aggag and Revett in their work on affective gaming based on GSR, have developed a basic first-person shooter (FPS) that was supposed to be played in two different interleaved difficulty levels ~\cite{aggag2011affective}. They considered players' stress level as a function of the difficulty of the game. They synchronously recorded players' GSR response and then mapped this signal to what happened during the game as difficulty level was manipulated. During the experiment they set the difficulty level randomly such that players all experienced the same distribution but not presentation of difficulty. Their principal idea was to acquire the score during boring and challenging play periods in order to see if there was any difference that could be attributed to level of difficulty ~\cite{aggag2011affective}.

They observed that the game did induce feelings of stress at the same time points during the play through players self report. The players' GSR signal that was recorded during play was pooled according to difficult/non-difficult regions and the data was analyzed with respect to the frequency and amplitude of the responses throughout the two phases of the game for each phasic response. Their result indicate that during the stressful periods (higher difficulty level), the skin conductance level increase and the frequency of the spontaneous GSRs increased (from 0.5 to 2.3 per minute on average). Aggag and Revett hoped to use the recorded GSR signal to provide subjects with a balance between basic and advanced play, by feeding back GSR level through the game logic to manage the affective state of the player ~\cite{aggag2011affective}.

Aggag and Revett could not determine if level of arousal had any effect on players' score, as a reflection of player performance, but that the affective state of the player can influence performance. In their study, increased difficulty level corresponded to increased score (performance). While they find it seemingly a counter-intuitive result, they suggest it should be due to increased engagement of the player which in turn may enhance their overall sensitivity to audio-visual stimuli and enhanced their reaction time. However, due to the limitations of their study they refused to draw a strong conclusion in this regard ~\cite{aggag2011affective}.

%% Affective gaming: measuring emotion through the gamepad.pdf
\subsection{Emotion and Standard Game Input Devices}
Sykes and Brown in their work on measuring emotion through a gamepad ~\cite{sykes2003affective}, both from a marketing perspective and also targeting current generation of video-games and available gaming technologies, suggesting to use current video game technologies to measure affect rather than introducing new equipment. They used modern game consoles' controller analogue buttons which indicate the pressure used when playing a game. The possibility of detecting a person's emotion through finger pressure ~\cite{clynes1977sentics}, makes the analogue buttons on the gamepad a possible resource for collecting data.

In their study, Sykes and Brown showed that data from gamepad preasure correlates with a player's level of arousal during game play. They developed a variant of the classic arcade game `Space Invaders' (Figure ~\ref{fig:space-invaders}) for their study. Players needed to shoot alien spacecraft as they march down the screen toward them. It was possible for the players to move to their left or right to avoid offensive attacks. They could also return fire by pressing a button on the gamepad. They have employed three levels of difficulty to change the players' level of arousal: easy, medium and hard. For the medium level the alien craft would march twice as fast, and the player would have the benefit of only two barriers. In the hard level the tempo of the alien craft was increased by a further factor of two, and the barriers were removed completely ~\cite{sykes2003affective}. Players have played different levels in random order and the amount of pressure exerted by the player on each button press has been recorded by the game.

Although Sykes and Brown in their study do not investigate the effect of NPC and environmental factors separately but based on their results, they conclude it is possible to determine the level of a player's arousal by the pressure they use when controlling the gamepad.

\img
{Space Invaders}
{Space Invaders - The original game used by Sykes and Brown}
{space-invaders.png}
{space-invaders}

%% Dynamic Difficulty Adjustment by Facial Expression.pdf
\subsection{Difficulty Level and Facial Expression}
Xiang et al. ~\cite{xiang2013dynamic} in their study on dynamic difficulty adjustment by facial expression provided an emotion based dynamic game adjusting prototype named Emotetris, which utilizes facial expression captured using a camera to assign the emotional state of the player to frustrated, relaxed, excited or bored. Their prototype adjusts game difficulty level dynamically according to these emotional states. Their method of dynamic adjustment combines the in-game performance and facial expressions of players to dynamically adjust the game difficulty. In their study they have shown how better the dynamic difficulty adjustment can attract players' attention when they were bored and release the pressure when they were frustrated.

They have adjusted Tetris to evaluate the performance of player. In their prototype the speed of dropping items is the parameter to be adjusted as it directly affects players. Participants preferred the facial expression adaptation to standard performance based adaptation.

% Affective game engines: motivation and requirements.pdf
% Biofeedback game design: using direct and indirect physiological control to enhance game interaction.pdf
% Boredom, engagement and anxiety as indicators for adaptation to difficulty in games.pdf
% A semantic generation framework for enabling adaptive game worlds.pdf
% Using gameplay semantics to procedurally generate player-matching game worlds.pdf
% Dynamic Difficulty Adjustment in Computer Games Through Real-Time Anxiety-Based Affective Feedback.pdf
% Dynamic Game Balancing by Recognizing Affect.pdf
% Pervasive Game Flow - Understanding Player Enjoyment in Pervasive Gaming.pdf
% The Influence of Implicit and Explicit Biofeedback  in First-Person Shooter Games.pdf
% The use of video game achievements to enhance player performance, self-efficacy, and motivation.pdf
% User Performance Tweaking in Videogames - a Physiological Perspective of Player Reactions.pdf
% Using Frustration in the Design of Adaptive Videogames.pdf
% Emotionally Adapted Games - An Example of a First Person Shooter.pdf

% TODO: talk about the following as a problem of this work
% However, how to change the game parameters in a certain situation is due to the custom of different players.
