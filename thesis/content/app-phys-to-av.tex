
The following 22 rules were used as described in Section ~\ref{subsec:fuzzi} to transform physiological variables into arousal-valence space:

\begin{enumerate}
\item If (GSR is high) then (arousal is high)
\item If (GSR is mid-high) then (arousal is mid-high)
\item If (GSR is mid-low) then (arousal is mid-low)
\item If (GSR is low) then (arousal is low)
\item If (HR is low) then (arousal is low)
\item If (HR is high) then (arousal is high)
\item If (GSR is low) and (HR is high) then (arousal is midlow)
\item If (GSR is high) and (HR is low) then (arousal is midhigh)
\item If (EMGfrown is high) then (valence is very low)
\item If (EMGfrown is mid) then (valence is low)
\item If (EMGsmile is mid) then (valence is high)
\item If (EMGsmile is high) then (valence is very high)
\item If (EMGsmile is low) and (EMGfrown is low) then (valence is neutral)
\item If (EMGsmile is high) and (EMGfrown is low) then (valence is very high)
\item If (EMGsmile is high) and (EMGfrown is mid) then (valence is high)
\item If (EMGsmile is low) and (EMGfrown is high) then (valence is very low)
\item If (EMGsmile is mid) and (EMGfrown is high) then (valence is low)
\item If (EMGsmile is low) and (EMGfrown is low) and (HR is low) then (valence is low)
\item If (EMGsmile is low) and (EMGfrown is low) and (HR is high) then (valence is high)
\item If (GSR is high) and (HR is mid) then (arousal is high)
\item If (GSR is mid-high) and (HR is mid) then (arousal is mid-high)
\item If (GSR is mid-low) and (HR is mid) then (arousal is mid-low)
\end{enumerate}
