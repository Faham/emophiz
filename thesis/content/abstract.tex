% Abstract less than or equal to 1 page
% one page stating what the thesis is about
% highlight the contributions of the thesis

Innovations in computer game interfaces continue to enhance the experience of players. Affective games –- those that adapt or incorporate a player's emotional state –- have shown promise in creating exciting and engaging user experiences. However, a dearth of systematic exploration into what types of game elements should adapt to affective state leaves game designers with little guidance on how to incorporate affect into their games. We created an affective game engine, using it to deploy a design probe into how adapting the player's abilities, the enemy's abilities, or variables in the environment affects player performance and experience. Our results suggest that affectively adapting games can increase player arousal. Furthermore, we suggest that reducing challenge by adapting non-player characters is a worse design choice than giving players the tools that they need (through enhancing player abilities or a supportive environment) to master greater challenges.
