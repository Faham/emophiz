%8 to 20 pages around 50 to 70 ref
%More than a literature review
%Organize related work - impose structure
%Be clear as to how previous work being described relates to your own.
%The reader should not be left wondering why you've described something!!
%Critique the existing work - Where is it strong where is it weak? What are the unreasonable/undesirable assumptions?
%Identify opportunities for more research (i.e., your thesis) Are there unaddressed, or more important related topics?
%After reading this chapter, one should understand the motivation for and importance of your thesis
%You should clearly and precisely define all of the key concepts dealt with in the rest of the thesis, and teach the
%reader what s/he needs to know to understand the rest of the thesis.

%%% 1 page: overview of what is going on

Using emotional responses to adapt interaction with a real-time play technology requires a method of identifying specific emotion states within an emotional space. Methods of describing emotions in the psychology literature include: basic emotion theory ~\cite{ekman1992argument, ekman1992there}, which uses a series of semantic labels (e.g., joy, fear) to identify discrete emotion categories; and dimensional emotion theory ~\cite{lang1995emotion, russell1989affect}, which argues that emotions reside in a two-dimensional space defined by arousal and valence. Regardless of how we characterize emotional response in a person, our goal is to sense the emotional state of a user and use that information in a real-time manner to adapt gameplay. Thus, we refer to a player having an affective state and we aim to adapt to a player's affect. The use of `affect' throughout this thesis reflects that we are less concerned with advancing the theories of emotion and rather are more concerned with using emotionally-relevant player states to drive gameplay.

In this chapter, research related to this thesis is presented. We start by introducing and reviewing common terminology used in the research on affect and emotion and the methods that have been used to measure affect and emotion.


\section{Affect and Emotion}

This section introduces common terms used in the literature along with different ways these terms are described.

\subsection{Terminology}
The terms \textit{affect} and \textit{emotion} are often used interchangeably and using these terms without any specific description highlighting their differences can be confusing. To avoid this confusion, it is important to understand the distinction between these terms. In this thesis, \textit{affect} is used in a more general sense that encompasses emotions ~\cite{forgas1995mood}, whereas \textit{emotions} are usually reactionary feelings often triggered by some particular physical or cognitive cause and are short in duration; individuals are usually aware of the presence of an emotion ~\cite{paiva2007affective} as emotion can be described as the conscience experience of affect.

Classical attempts to describe emotion can be categorized into two major approaches: those that try to describe emotion by emphasizing its cognitive (mental) aspects and those that concentrate on its bodily (physical) aspects. Walter Cannon is usually credited for the cognitive approach by having suggested that emotion is an experience within the brain, independent of the sensations of the body ~\cite{cannon1927james}. On the other hand the physical approach has largely been attributed to William James, who suggested that physiological responses (e.g. elevated heart rate) are the center of focus that occurs just prior or during an emotional episode ~\cite{paiva2007affective}.

In more recent approaches, emotion has been considered as a combined result of cognitive and physiological changes simultaneously ~\cite{paiva2007affective}. Body chemistry changes and thoughts can both contribute to the definition of emotions – Schachter suggests that emotion is our interpretation of a specific physiological reaction along with our mental situation, and that we labeled this as an emotion (e.g. fear) ~\cite{schachter1964interaction}. In this thesis, \textit{emotional state} refers to the combinational internal dynamics (both cognitive and physiological) that are perceived by an individual during an emotional experience ~\cite{paiva2007affective}.

\section{Describing Emotion}

The two main ways of identifying emotions in related research is by dividing them into discrete categories or assuming a continuous dimensional space in which emotions can be defined.

\subsection{Discrete Categories}
The discrete approach – also known as the basic emotion theory – largely relies on language in its mission to describe emotion; in fact, it begins by identifying specific labels people attach to different emotional episodes and then suggests categories of emotions. Examples of such labels (or categories) include excitement, anger, fear, sadness and happiness. However, the suggested discrete categories in the categorical approach do not necessarily agree with one another. Relying on language to describing emotions not only led suggested categories to vary across languages, but also within a language. The variability and disagreement in the literature suggests a lack for clear definitions or boundaries for these states, which has caused difficulties when comparing different research approaches. In-availability of specific categories in other languages also makes research using this approach difficult ~\cite{zimmermann2006extending}.

Recent work on basic emotion theory identifies anger, disgust, fear, happiness, sadness, and surprise ~\cite{peter2006emotion} as the concise set of primary emotions. These are actually the smallest set of universal categories researchers agreed upon by researchers ~\cite{zagalo2004story}. The discrete approach also claims that these primary emotions are distinguishable from each other and other affective phenomena ~\cite{dalgleish1999handbook}.

\subsection{Continuous Dimensions}

The dimensional emotion theory argues that all emotional states reside in a two-dimensional space, defined by arousal and valence. This approach - described by Russell in ~\cite{russell2003core} - introduces the idea of core affect to identify emotions. It holds core affect accountable for feelings triggered by specific events and describes it as being composed of two independent dimensions: arousal and valence. Figure ~\ref{fig:russel-av-space} illustrates the concept of arousal and valence space describing various emotions known as common emotion categories.

\img
{Russell's arousal and valence model}
{Russell's circumplex model with two axes of arousal and valence \footnotemark.}
{russell-av-space.pdf}
{russel-av-space}
\footnotetext{Photo credit: http://imagine-it.org/gamessurvey/}

The energy or the degree of activation of an individual (which brings with it a sense of mobilization) is usually referred to as \textit{arousal}. The arousal state is the physiological and psychological state of being reactive and responsive to a stimuli. The flight-or-fight response, as introduced in Cannon's theory ~\cite{stern2001psychophysiological} is a physiological reaction that occurs in response to a perceived threat or stimuli and focuses on the physiological changes that occur in the body during these situations. Different qualities of arousal are usually studied as low (e.g. sleepiness) to high (e.g. excitement).

Valence as used in the study of emotions, means the intrinsic attractiveness (positive valence) or aversiveness (negative valence) of an event or situation ~\cite{frijda1986emotions}. However in many related studies of emotion, the term is also used to identify popular emotions by their negative or positive impressions. Emotions with lower valence are those that are less desired such as anger and fear, and emotions with higher valence are those that are more desired such as joy and happiness.

Lang used a 2-D space defined by arousal and valence (pleasure) (AV space) to classify emotions ~\cite{lang1995emotion}. Valence is described as a subjective feeling of pleasantness or unpleasantness while arousal is the subjective state feeling activated or deactivated ~\cite{barrett1998discrete}. Using an arousal-valence space to create the Affect Grid, Russell believed that arousal and valence are cognitive dimensions of individual emotion states. Affect is a broad definition that includes feelings, moods, sentiments etc. and is commonly used to define the concept of emotion ~\cite{picard2003affective}. Russell's model has two axes that might be labeled as displeasure/pleasure (horizontal axis) and low/high arousal (vertical axis) It is not easy to map affective states into distinctive emotional states, however these models can provide a mapping between predefined states and the level of arousal and valence ~\cite{zagalo2004story}, Figure ~\ref{fig:russel-av-space}.

\section{Recognizing Emotions}

While there are various opinions on identifying emotional states, classification into discrete emotions ~\cite{dalgleish1999handbook}, or locating emotions along multiple axes ~\cite{russell1989affect, lang1995emotion}, both had some success in using physiology to identify emotional states ~\cite{cacioppo2000psychophysiology}.

In this thesis, both the categorical and dimensional approaches are used for developed models. The model developed for capturing emotional state responses is coupled with gathered subjective emotional experiences of our participants based on a categorical approach. Using a categorical approach when collecting emotional experiences subjectively is the most practical method, as it is far easier for participants to communicate in a language that they can understand (emotional categories rather than the degree of arousal or valence) to describe their emotional state best. However although we did not want to use a data collection process that required the participants to learn new terminologies and describe their emotional state with unfamiliar terms, participants were introduced to the concepts of arousal, valence and dominance. Given example emotions for different levels of these variables, participants described their affective state by choosing images based on these concepts. The developed model for the affect space uses the dimensional model as in Figure ~\ref{fig:russel-av-space} to provide a mapping between the original emotional categories and a dimensional space. These models are further elaborated on in Chapter ~\ref{chap:impl}.

Both mentioned models for identifying emotions convey some practical issues in emotion measurement. In an HCI context, the stimuli for potential emotions may vary less than in human-human interaction (e.g., participant verbal expressions and body language) ~\cite{zhang2010service} and also the combination of evoked emotions ~\cite{peter2006emotion}. However with help of physiological signals and the fuzzy logic model we use, such issues with our dimensional emotion models will be minimized. Though it is anticipated that we will observe different ranges of evoked emotions while interacting with play technologies compared to interacting with other humans in daily life ~\cite{zhang2010service}. Our dimensional emotion models also suffers some other problems. One problem is that arousal and valence are not independent and one can impact the other ~\cite{mandryk2007fuzzy}. Continuously capturing emotional experiences in this applied setting raises other problems. Subjective measures based on dimensional emotion theory, such as the Affect Grid ~\cite{russell1989affect} and the Self-Assessment Manikin ~\cite{bradley1994measuring}, allow for quick assessments of user emotional experiences but they may aggregate responses over the course of many events ~\cite{zhang2010service}.

There are many visible features that can be observed and measured in our everyday interactions for consideration as emotional indicators. Different emotional indicators that have been studied to determine affect include facial expressions, gestures, postures, language, pressure, and pupil dilation ~\cite{picard2003affective}. Facial expressions for example can help us to figure out whether someone is distracted, frustrated or happy. Researchers have created sophisticated face-tracking software to analyze facial expressions in order to find out emotional state of the user ~\cite{partala2006real, sebe2006emotion}. Some researchers have extended this work by identifying facial points that undergo significant thermal changes with a change in expression and thus have performed person-independent classification to do affect interpretation using infrared measurement of facial skin temperature variations ~\cite{khan2006automated}. Other recent work has pushed the borders even further by using observable facial features that are only visible to machines. Work by Takano et al., for example, has shown how to measure heart rate based on a partial average image brightness of the subject's skin using consecutively captured time-lapsed images ~\cite{takano2007heart}.

Many physiological changes that occur in the body during an emotional episode are not visible to another person. Many researchers have considered using physiological data to identify emotional states. It was first speculated by William James to use patterns of physiological responses to recognize emotion ~\cite{cacioppo2000psychophysiology}. Although this approach does not consider the individual's psyche and state of mind to identify emotions, evidence suggests that physiological data sources can differentiate among some emotions ~\cite{ekman1983autonomic}. Picard et al. performed a feature-based recognition of eight emotional states from GSR, EMG of the jaw, BVP and respiration over multiple days ~\cite{picard2001toward}. Their work presents and compares multiple algorithms for feature-based recognition of emotional states partially corrected for day-to-day differences and provides an 81\% accuracy for recognizing eight emotional states. Mandryk et al. showed how to measure and use physiological metrics such as galvanic skin response (GSR), respiration, electrocardiography (EKG), and electromyography of the jaw (EMG) as indicators of participants' affective states while playing video games ~\cite{mandryk2007fuzzy}.

\section{Measuring Affect}
When evaluating affective interfaces and interactions in HCI, one of the most important and primary challenges is to detect the affective state of the user. Measuring affect can be addressed under different titles such as sensing, detection or recognition. However, we chose to use `measurement' to signify all these different expressions. There are multiple ways that researchers measure affect in people. For example, researchers have used facial expressions ~\cite{partala2006real}, typing rhythms ~\cite{epp2011identifying}, and voice signal analysis ~\cite{picard2003affective} to characterize a user's affective state. However, the most common approach is to gather physiological signals and use mathematical modeling approaches to characterize affective state reflected by the physiological measurements ~\cite{mandryk2007fuzzy}. For example, heart rate (HR), blood pressure, respiration, galvanic skin response (GSR), and facial EMG (Electromyography) are physiological variables that have been shown to correlate with various affective states ~\cite{mandryk2008physiological}. Interpreting physiological measures can be difficult, due to noisy signals and difficulties with inference; however, recent progress in this area has been promising. In addition, there has been work to apply physiological affect recognition approaches to video game evaluation. Mandryk and Atkins presented a method of continuously identifying affective states of a user playing a computer game ~\cite{mandryk2007fuzzy}. Using the dimensional emotion model and a fuzzy logic approach on a set of physiological measures, the authors transform GSR, HR, and facial EMG (for frowning and smiling) into arousal and valence variables and then transform arousal and valence variables into five player-centric affective states including: boredom, challenge, excitement, frustration and fun. The advantage of continuously and quantitatively assessing user's affective state during an entire play session using their fuzzy logic model is what makes their model appropriate for real-time play technologies. Classically there are two major approaches for affect measurement: physiological measures and self-report. In the following sections, we present a brief description of various self-report approaches and continue with a look into today's most popular physiological measures for measuring emotion.

\subsection{Self-Report}
Self-report measures classify the emotional state of an individual by directly questioning them. This is usually done through a familiar language and vocabulary, or sometimes by using images that carry a common meaning within different languages and cultures. This is in fact trying to find out about an individual's emotional state through his or her verbal descriptions, and it can have different forms like rating scales, standardized checklists, questionnaires, semantic graphical differentials and projective methods. Self-report is maybe the simplest and easiest way to approach the issue of affect measurement, and it suffers some major weaknesses. Criticisms of self-report methods include the possibility that they draw attention to what the experimenter is trying to measure, that they fail to measure mild (low intensity) emotions, and that they are lack construct validity ~\cite{isen2007some}.

\subsubsection{Game Engagement/Experience Questionnaire}
The Game Engagement/Experience Questionnaire (GEQ) measures a gamer's engagement during video game play ~\cite{brockmyer2009development}. This questionnaire consists of 19 items scored on a Likert scale. This questionnaire specifically measures engagement level as absorption, flow, presence and immersion. Cronbach's alpha for the current 19-item version of the GEQ is .85. The Rasch estimate of person reliability (the Rasch analog to Cronbach's alpha) for the 19-item version is .83 and the item reliability is .96 ~\cite{brockmyer2009development}. In this work, this questionnaire is used as a subjective measure in Chapter ~\ref{chap:exprm}.

\subsubsection{Intrinsic Motivation Inventory}
The Intrinsic Motivation Inventory (IMI) utilizes several sub-scales that relate to user experience during a targeted activity ~\cite{ryan1983relation}. This questionnaire is a useful measure for interactive technologies such as games and has been utilized in several studies. For this study, the Interest-Enjoyment sub-scale that contains 5 questions, the Effort sub-scale that contains 4 questions, and the Pressure-Tension sub-scale that contains 4 questions was used. The interest-enjoyment sub-scale is associated with self-reported intrinsic motivation. More information about this questionnaire and the experiment can be found in chapter ~\ref{chap:exprm}.

\subsubsection{Player Experience of Need Satisfaction}
Player Experience of Need Satisfaction model (PENS) introduces a practical theory of player motivation that has meaningfully contributed to developers' understanding of what really satisfies players. This work done by Immersyve ~\cite{rigby2007pens} provides a practical testing methodology and analytic approach with proven value. Numerous data demonstrate competence, autonomy and relatedness at the heart of player's enjoyment of games and how games are valued, PENS outlines and measures these three intrinsic psychological needs through 21 items scored on a Likert scale ~\cite{rigby2007pens}. The PENS model can significantly predict positive experiential and commercial outcomes through collecting data on how these needs are being satisfied, in many cases this has happened much more strongly than more traditional measures of fun and enjoyment. It is important to note the plausible predictive values demonstrated by PENS model repeatedly have been done regardless of genre, platform or even the individual preferences of players ~\cite{rigby2007pens}.

\subsubsection{Self-Assessment-Manikin Arousal Scales}
The Self-Assessment Manikin (SAM) ~\cite{lang1985cognitive} presents a promising solution to the problems that have been associated with measuring emotional response in Mehrabian and Russell's three emotional dimensions or pleasure (valence), arousal, and dominance ~\cite{russell1977evidence}. SAM takes a visual approach to design an alternative to the sometimes-cumbersome verbal self-report measures ~\cite{lang1985cognitive}.

\img
{SAM The Self-Assessment Manikin}
{The Self-Assessment Manikin}
{sam-russell.png}
{sam-russell}

SAM has been used in numerous psychophysiological studies since its development. The correlations between scores obtained using SAM and those obtained from Mehrabian and Russell's semantic differential procedure were impressive for both pleasure (.94) and arousal (.94) and smaller but still substantial for dominance (.66) ~\cite{lang1985cognitive}. Similar results were found by Morris and Bradley ~\cite{morris1995observations} through a SAM evaluation of 135 emotion adjectives that were factor analyzed by Mehrabian and Russell.

By using visually oriented scales and a graphic character, it is clear that SAM eliminates the majority of problems associated with verbal measures or nonverbal measures that are based on human photographs. The simple and visual scales help individuals complete ratings on the SAM scales in under 15 seconds, and therefore this allows numerous stimuli to be tested in a short amount of time and may cause less respondent fatigue than the verbal measures. Experiment participants have expressed greater interest in SAM ratings versus verbal self-reports in a number of studies and have stated that SAM is more likely to hold their attention ~\cite{lang1985cognitive}. A third advantage is that both children and adults readily identify with the SAM figure and easily understand the emotional dimensions it represents ~\cite{lang1985cognitive}. Because SAM is a culture-free, language-free measurement, it is suitable for use in different countries and cultures ~\cite{bradley1993affective}.

There is longstanding tension in evaluation research between the `objective' and the `subjective' approaches. In the objective approach the focus is on measuring `hard' facts such as players performance in terms of in-game statistics (e.g. number of killed enemies or collected points), whereas on the other hand, the subjective approach considers `soft' matters such as gamers' satisfaction with the play experience or players' experience of flow in the game. The objective approach roots in the tradition of social statistics, which dates back to 19th century. The subjective approach stems from survey research, which took off in the 1960's ~\cite{veenhoven2002social}.

\subsection{Physiological Measures}
Physiological signals such as facial expressions, vocal tone, skin conductance, heart rate, blood pressure, respiration, pupillary dilation, electroencephalography (EEG) or muscle action, are being used to determine the intensity and quality of and individual's internal affective state, and are usually referred to as physiological measures. As for self-report measures, there are concerns with physiological measures that usually relate to first,the setup, invasiveness, and attendance that the involved devices require, and second, the association of specific physical responses with a particular type of emotion because of individual variability ~\cite{depaula2005cognitive}.

In the next sections, a number of the most popular physiological measures that are also used in this thesis are introduced.

\subsubsection{Galvanic Skin Response}
Skin conductance, also known as galvanic skin response (GSR) or electrodermal response (EDR), is a method of investigating electrical conductance of the skin. This feature varies depending on the moisture of the skin due to sweat. The fact that sweat is controlled by the sympathetic nervous system ~\cite{seiger2002essentials} makes this measure quite helpful to investigate the affective state of an individual. In other words, skin conductance can be used as an indication of physiological arousal. The sweat gland activity in certain areas of the skin, such as finger tips, is largely dependent to the sympathetic branch of the autonomic nervous system. For example, it would be increased if the person was highly aroused and therefore, skin conductance would change. Thus, skin conductance is a good measure of emotional and sympathetic responses ~\cite{carlson2013physiology}.

Galvanic skin response can be measured by looking at changes of galvanic skin resistance and galvanic skin potential. Galvanic skin resistance refers to measured electrical resistance between two electrodes while a weak current is passing through them. These electrodes are usually placed on certain areas of skin about an inch apart. Galvanic skin potential is the measured voltage between two electrodes while no external current being applied. This potential is measured by connecting electrodes to voltage amplifiers. The recorded resistance and voltage varies dependent on the emotional state of the subject ~\cite{pflanzer2013galvanic}.

\img
{Galvanic Skin Response (GSR) Sensor}
{Galvanic Skin Response (GSR) Sensor}
{gsr-sensor.png}
{gsr-sensor}

\img
{Galvanic Skin Response (GSR) Signal}
{Galvanic Skin Response (GSR) Signal ~\cite{wiki2014gsr}}
{gsr-signal.pdf}
{gsr-signal}

The relation between sympathetic activity and emotional arousal due to a stimuli can be easily detected through the response of the skin. The subtle changes in skin conductance, when the device is correctly calibrated, can be measured and rationalized. Though identification of particular specifications of the emotional episode merely by looking at these skin conductance changes seems to be impossible ~\cite{pflanzer2013galvanic}.

\subsubsection{Heart Rate}
The easiest way to measure heart rate is by finding the pulse of the heart by looking at any region of body where the artery's pulsation is easily detectable at the surface of skin. By pressuring that region with the index and middle fingers against the underlying structures, such as bone, the pulse of the heart can be detected. The neck under the corner of the jaw, the wrist and the upper arm are the best places to find the blood vessels close to the skin's surface and therefore easily feel the pulse of the heart when blood is pumped through the body. %todo: add refrence

Electrocardiograph or ECG (also abbreviated EKG) is the device usually used for more precise determination of the heart's pulse. This device is quite popular in clinical settings for continuous monitoring of heart, particularly in critical care settings such as ICU. EKG uses electrodes placed on the surface of the skin to measure the electrical activity of the heart. Usual places to attach these electrodes are on the chest, forearm or legs. Conductive gels should be applied on the bare skin before attaching these electrodes, also there should be no gap between the electrodes and the skin so the area usually needs to be shaved and must be free of hair to prevent interferences with the sensors ~\cite{stern2001psychophysiological}.

On an ECG, the heart rate is measured using the R wave to R wave interval (RR interval). Accurate R peak detection is essential in signal processing equipment for heart rate measurement ~\cite{pise2011thinkquest}. In this thesis, this has been done by looking at signal derivatives after applying smoothing passes to the signal data.

\img
{EKG RR Interval}
{EKG RR Interval ~\cite{wiki2014bvp}}
{ekg-rr-interval.pdf}
{ekg-rr-interval}

A Blood Volume Pulse sensor (BVP or photoplethysmograph) or Pulse oximetry are  comparatively non-invasive methods for monitoring an individual's pulse. In BVP, an infra-red beam in bounced against a skin surface and measures the pulse by looking at the amount of reflected light. The reflected amount of light would change by passing through a different volume of blood in the skin. Therefore when there is a larger volume of blood in the skin, its red color causes it to absorb larger amount of other colors and more red color is reflected, but when the skin does not contain large volumes of blood, more amounts of other colors are reflected. %todo: add ref
Using the BVP signal in addition to the heart rate, the software can usually also calculate the inter-beat interval. The amplitude of the BVP deviation can also be a useful measure. Heart Rate Variability can also be calculated with the BVP.

\img
{Blood Volume Pulse (BVP) Sensor}
{Blood Volume Pulse (BVP) Sensor}
{bvp-sensor.png}
{bvp-sensor}

Heart Rate Variability (HRV) is the phenomenon of variation in the time interval between heartbeats and therefore the heart rate. It is measured by looking at variation in beat-to-beat interval. HRV is an interesting measure to look at in the field of psychophysiology. HRV is usually correlated to emotional arousal. Schwarz et al. have shown that hopelessness is associated with decreased heart rate variability during championship chess games ~\cite{schwarz2003hopelessness}. Ivarsson et al. were able to show during violent (vs. nonviolent) gaming, there was a significantly higher activity of the very low frequency component of the HRV and total power ~\cite{ivarsson2009playing}. In their research they compared the player experience in the violent game - Manhunt (Rockstar Games, 2004) - with the nonviolent game Animaniacs (Ignition Entertainment, 2005). In Manhunt the player is a murderer, sentenced to death and his only chance to survive is to kill everyone he meets by beating and kicking. He should use simple weapons available like plastic bags and baseball bats stolen from murdered people. The game takes place in an abandoned area where criminals dwell during night time. It is presented in a detailed and naturalistic fashion. In Animaniacs, the game occurs during day time, and characters and surroundings give a cartoon-like impression. Ivarsson et al. concluded that analyzing HRV seems to be a useful approach for studying the impact of violent content in video games ~\cite{ivarsson2009playing}.

\subsubsection{Facial Electromyography}

Electromyography in general refers to a technique that measures muscle activity by detecting and amplifying the tiny electrical impulses that are generated by muscle fibers when they contract. Facial Electromyography (fEMG) primarily focuses on two major muscle groups in the face. The corrugator supercilii group, which is usually associated with frowning and the zygomaticus major muscle group, which is associated with smiling ~\cite{larsen2003effects, sato2008enhanced}.

Many studies have assessed Facial EMG's utility as a tool for measuring emotional reaction ~\cite{dimberg1990facial}. Studies have found that activity of the corrugator muscle, which lowers the eyebrow and is involved in producing frowns, varies inversely with the emotional valence of presented stimuli and reports of mood state. Activity of the zygomatic major muscle, which controls smiling, is said to be positively associated with positive emotional stimuli and positive mood state.

\img
{Facial muscles associated with frowning and smiling}
{On left side: Corrugator supercilii muscle (associated with frowning), on right side: Zygomaticus major muscle (associated with smiling) ~\cite{wiki2014facial}}
{emg-facial-muscles.png}
{emg-facial-muscles}

In many research, facial EMG has been utilized as a technique to recognize and track positive and negative emotional reactions to a stimulus as they occur ~\cite{wolf2005facial}. A large number of those experiments have been conducted in controlled laboratory environments using a range of stimuli, e.g., still pictures, movie clips and music pieces.

In 2012, Durso et al. were able to show that facial EMG could be used to detect confusion, both in participants who admitted to being confused and in those who did not, suggesting that it could be used as an effective addition to a sensor suite as a monitor of loss of understanding or loss of situation awareness ~\cite{durso2012detecting}.

In gaming and Human-Computer Interaction (HCI) - Ravaja ~\cite{ravaja2008psychophysiology}, Hazlett ~\cite{hazlett2006measuring} and Mandryk ~\cite{mandryk2007fuzzy} used facial EMG techniques to demonstrate that positive and negative emotions can be measured in real time during video game play. The emotional profiling of games give a useful evaluation of a game's impact on a player, how compelling they find the game, how the game measures up to other games in its genre, and how the different elements of the game enhance or detract from the game's approach to engaging the player ~\cite{nacke2010affective}.

One of the major problems with using physiological devices to measure affect is the intrusive nature of the technology. Although physiological sensors can provide lots of useful data about the user in the course of interaction, it is usually quite limiting to use sensors in many ways. Sensors usually need special attention in terms of their placement and connection to the target, particularly because the target is sometimes moving. Some sensors are inherently sensitive to movement and might generate a large amount of noisy signals, which need to be detected and filtered out by the software analyzing the signal. On the other hand, some of the sensors (such as the respiration sensor) can hardly be designed for realistic casual interactions. Furthermore, the presence of an unusual device attached to the user might itself have some influence on the user's emotional experience.

There are some physiological approaches that let us detect affect states with fewer limitations. Wireless and wearable devices or even devices with no need to have any contact with the participants – such as thermal cameras that identify increased blood flow in particular regions of the skin – are of this category ~\cite{puri2005stresscam}. However in the case of thermal cameras, this technology – although not as obtrusive as other physiological approaches such as GSR sensors – still requires a relatively expensive device that is not usually found in typical computer settings. This main drawback of expensive technologies is still typical of many other physiological sensors such as GSR sensors. The requirement for such expensive specialized equipment limits the applicability of widespread adoption of these sensors.
