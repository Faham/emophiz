% 5 to 10 pages
% talk about the experimentation
% adequacy, efficiency, productiveness, effectiveness
% (choose your criteria, state them clearly and justify them)
% be careful that you are using a fair measure, and that you are
% actually measuring what you claim to be measuring
% if comparing with previous techniques those techniques
% must be described in Chapter 2
% be honest in evaluation
% admit weaknesses

\section{Participants}
Data were recorded from 15 male and 2 female University students,
aged between 18 and 32 (M = 25.00, SD = 3.875). As part of the
experiment procedure demographic data were collected with special
respect to the suggestions made by ~\cite{?}. Of the participants 94.1\% were
right-handed. 41.2\% of participants rated their computer skills as Advanced
while the rest of 58.8\% rated their skills as Intermediate. 35.3\% of
participants have described themselves playing video games every day, while
41.2\% of them described themselves playing video games a few times per week
and 17\% have been playing video games a few times per month and the rest of
5.9\% have been playing video games a few times per year. All participants
have used PC as gaming system while 76.48\% of them also have used at least
one of the four popular console platforms (XBox360, PS3, PS2, Wii) for gaming.
All of participants had at least some experience with 3D shooting games like
First Person Shooters. 47.1\% have described themselves playing 3D shooting games
many times, while another 41.2\% described themselves as experts in
3D shooting games; Only a total of 11.8\% had limited or intermediate experience
with 3D shooting games. Among the participants only 5.9\% had intermediate
experience in using mouse in games, 35.3\% of them declared using mouse in games
many times and 58.8\% of them described themselves as experts in doing that.


\section{Design}
\section{Procedure}

\section{Materials and Measures}
\subsection{Game Experience Questionnaire}
\subsection{Self-Assessment-Manikin Valence Scales}
\subsection{Intrinsic Motivation Inventory}
\subsection{Player Experience of Need Satisfaction}
\subsection{Game Engagement Questionnaire}

This study has been run during three weeks, each session took about
45 minutes. 17 participants of moderate to expert FPS players were
asked to play the game under four different conditions: No
Adaptation, Player Adapted, NPC Adapted and Environment Adapted.
The order of played conditions was circulated between different players.
Participants did not know what different conditions exists and
which condition they are currently playing, at the start of the play
session, they were required to press one of the four buttons on the
entrance ramp labeled 1 to 4, and when any one of these buttons were
pressed, the Affect engine started calibrating players signals for 60
seconds, during the calibration mode, no adjustment no any of game
parameters is applied, no matter which condition is being played.
After the one minute of calibration the system decides the standard
range of signal for player’s excitement value. After that except for
the condition number 1 which is the no adaptation mode, the captured
excitement value is normalized in the calibrated player range of
excitement into a value between 0 and 1, and this value is then used to
adjust the game parameters, this process of capturing, adjusting and
applying the signal value would continue for 3 minutes until the
next cycle of calibrating and adjustment starts. The player is
required to play every condition for at least 5 minutes to ensure
capturing of a complete cycle of calibrating and adjustment.
After playing each condition, the player is asked to rest for 7
minutes, during this time the player is asked to fill out between
condition questionnaires, which tries to ask the participant to
self-estimate his affect level.
The Player Mode is labeled as condition number 2, the NPC mode is
labeled as condition number 3 and the Environment mode is labeled
as condition number 4. From the 17 participants in this study,
one has been lacking adequate level of expertise and therefore was
unable to continue doing the required tasks at the expected level
and therefore his results was not usable for this study.
The image of signal values for this participant is depicted at the
following:

An image of a regular participant signal values is depicted at
the following. In this image from left to right the light blue line
shows different conditions being played, and when the light blue
line is declining towards its base value, that is the period that
participant is asked to stop playing and instead relaxing and filling
out the questionnaires. The blue line is the normalized GSR signal
value of the participants which is used as an estimation of his
excitement level. The yellow green and pink lines are showing the
three different modes of Player, NPC and Environment parameters
being adapted

Following image is the GSR signal of players playing different
conditions from 1 to 4. From left to right the conditions are the
Default, Player, NPC and the Environment mode. This signals are all
based to an initial start value of 100, during the play experience,
some of them had gone bellow the start point and some other had risen
above that. Also the start time for each different condition is
shifted 500 seconds times the number of condition, from 0.

The following image is the average of GSR values for players in four
different conditions from left to right: Default, Player, NPC and
Environment modes
