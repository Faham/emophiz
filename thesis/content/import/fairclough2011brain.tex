The brain and body provide a wealth of information
about the physiological, cognitive and emotional state
of the user. There is increased opportunity to use these
data in computerised systems as forms of input control.
As entry level physiological sensors become more widespread,
physiological interfaces are liable to become
more pervasive in our society (e.g., through mobile
phones). While these signals offer new and exciting
mechanisms for the control of interactive systems, the
issue of whether these physiological interfaces are appropriate
for application and offer the user a meaningful
level of interaction remains relatively unexplored.
------------------------
Physiological interfaces are currently undergoing a
boom in popularity as sensor technology is becoming
cheaper and more convenient to use (i.e., wearable).
Therefore, brain and body interactive systems are being
experimented with in a wide range of different application
domains, from health and fitness to entertainment
and self-experimentation [4,6]. Physiological interaction
has traditionally been used as medical tool, for
example as an alternative form of input control for disabled
persons or a form of psychological state management
(i.e., biofeedback therapies). Of late physiological
interfaces have focused on exploring what benefits
these types of interactions offer healthy people.
For example: in task performance [1] and entertainment
[2,6]. Physiological interactive systems (or
computers) are still in their explorative phase, as prior
limitations in sensing technologies have limited the deployment
of physiological computers to controlled environments
[5]. However, since sensors become less of
an issue, it is easier to explore what type of user experiences
can be facilitated with physiological signals.
------------------------
