% 5 to 10 pages
% Thesis Statement (one or two sentences)
% What is your thesis about and what have you done?
% If you have a hypothesis what is it?
% How will you test (prove/disprove) your hypothesis?
% Motivation
% Why is this problem you've worked on important
% Goals / Objectives
% What are you trying to do and why?
% How will you or the reader know if or when you've met your objectives?
% **** Contributions *****
% What is new, different, better, significant?
% Why is the world a better place because of what you've done?
% What have you contributed to the field of research?
% What is now known/possible/better because of your thesis?
% Outline of the thesis (optional)

Since computers are playing a significant role in our daily life, the need for a more friendly and natural communication interface between human and computer has continiously increased. Making computers capabale of perceiving the situation in terms of most human specific factors and responding dependent to this perception is of major steps to acquire this goal. If computers could recognize the situation the same way as human does, they would be much more natural to communicate. Emotions are of important and mysterious human attributes that have a great effect on people's day to day behavior. Researchs from neuroscience, psychology, and cognitive science, suggests that emotion plays critical roles in rational and intelligent behavior ~\cite{picard2001toward}. Apparently, emotion interacts with thinking in ways that are nonobvious but important for intelligent functioning ~\cite{picard2001toward}. Scientists have amassed evidence that emotional skills are a basic component of intelligence, especially for learning preferences and adapting to what is important ~\cite{mayer1993intelligence, goleman2006emotional} People used to express their emotions through facial expressions, body movement, gestures and tone of voice and expect others understand and answer to their affective state. But sometimes there is a distinction between inner emotional experiences and the outward emotional expressions ~\cite{picard2003affective}. Some emotions can be hard to recognise by humans, and inner emotional experiences may not be expressed outwardly ~\cite{jones2007biometric}. Recent extensive investigations of physiological signals for emotion detection have been providing encouraging results where affective states are directly related to change in inner bodily signals ~\cite{jones2007biometric}. However whether we can use physiological patterns to recognize distinct emotions is still a question ~\cite{picard2001toward, cacioppo1990inferring}.

Although the study of affective computing has increased considerably during the last years, few have applied their research to play technologies ~\cite{sykes2003affective}. Emotional component of human computer interaction in video games is surprisingly important. Game players frequently turn to the console in their search for an emotional experience ~\cite{rouse2010game}. There are numerous benefits such technology could bring video game experience, like: The ability to generate game content dynamically with respect to the affective state of the player, the ability to communicate the affective state of the game player to third parties and adoption of new game mechanics based on the affective state of the player ~\cite{sykes2003affective}. This work concentrates on developing a real-time emotion recognition system for play technologies which can quantify player instant emotional state during a play experience The rest of the paper is organized as follows: in Section 2 we outline different emotion recognition theories with an overview of physiology sensors. In Section 3 we demonstrate some implementation details of the system. We then describe the experimental setup in Section 4 before giving our results in Section 5. Finally, we give conclusions in Section 6.
